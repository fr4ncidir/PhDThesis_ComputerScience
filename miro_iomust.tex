\begin{table*}
\centering
\footnotesize
\caption{MIRO Report \cite{matentzoglu2018miro} of the IoMusT Ontology -- Part I of III}
\label{tab:iomust_miro1}
\begin{tabular}{p{.35\textwidth}p{.65\textwidth}}
\toprule
\multicolumn{2}{c}{\textbf{A. The basics}} \\
\midrule
\textbf{A.1 Ontology name} \textsc{must} &  Internet of Musical Things Ontology (IMTO), version 0.1 \\
\textbf{A.2 Ontology owner} \textsc{must}  & Francesco Antoniazzi \\
\textbf{A.3 Ontology license} \textsc{must}  & GNU General Public License v3.0 \\
\textbf{A.4 Ontology URL} \textsc{must} & \url{https://github.com/fr4ncidir/IoMusT/blob/master/iomust.owl} \\
\textbf{A.5 Ontology repository} \textsc{must}  & \url{https://github.com/fr4ncidir/IoMusT} \\
\textbf{A.6 Methodological framework} \textsc{must}  & The ontology defines the needed concepts to create a Musical Things IoT environment by introducing the namespaces \texttt{iot} and \texttt{iomust}, and connecting them to well known ontologies like \ontoref{sosa} and \ontoref{prov}. \\
\toprule 
\multicolumn{2}{c}{\textbf{B. Motivation}} \\  \midrule
\textbf{B.1 Need} \textsc{must} & The Internet of Musical Things is an unexplored field of IoT that at the moment lacks of semantic representation. \\
\textbf{B.2 Competition} \textsc{must} & At the moment, only with ontologies in IoT panorama. So far, no ontologies are available joining Music and IoT. \\
\textbf{B.3 Target audience} \textsc{must} & Developers of IoT applications applied to music. \\
\toprule
\multicolumn{2}{c}{\textbf{C. Scope, requirements, development community}} \\  \midrule
\textbf{C.1 Scope and coverage} \textsc{must}  & The ontology covers the concepts necessary to create a Musical Things IoT environment. The two namespaces identified in addition are extended, by plugging in references to other well known ontologies. The result is a complete vocabulary available to develop interoperable applications within and without the musical and artistic domain.
\\
\textbf{C.2 Development community} \textsc{must} & Advanced Research Center on Electronic Systems (ARCES) of the University of Bologna. Centre for Digital Music (C4DM), Queen Mary University of London. \\
\textbf{C.3 Communication} \textsc{must}  & \url{https://github.com/fr4ncidir/IoMusT/issues} \\
\midrule
\multicolumn{2}{c}{\textbf{D. Knowledge acquisition}} \\ \midrule
\textbf{D.1 Knowledge acquisition method} \textsc{must}  & Analysis of the available literature on Semantic Web, ontologies and IoT. In particular, how to represent music and musical instruments, devices and their components. Competency questions on the relevant domain.\\
\textbf{D.2 Source knowledge location} \textsc{should}  & Competency Questions \\
\textbf{D.3 Content Selection} \textsc{should}  & Things, Musical Things, Smart Things, Wearable Things: devices for IoT, applied to the collaborative production of musical content. \\
\toprule
\end{tabular}
\end{table*}

\begin{table*}
\centering
\footnotesize
\caption{MIRO Report \cite{matentzoglu2018miro} of the IoMusT Ontology -- Part II of III}
\label{tab:iomust_miro2}
\begin{tabular}{p{.35\textwidth}p{.65\textwidth}}
\toprule
\multicolumn{2}{c}{\textbf{E. Ontology content}} \\ \midrule
\textbf{E.1 Knowledge representation language} \textsc{must} & OWL 2 generated by Protégé v5.5.0beta; however, the ontology is at this stage only descriptive, and it uses a reduced subset of OWL 2 capabilities, being the Description Logic ALCRIF(D). \\
\textbf{E.2 Development environment} \textsc{optional} &  Prot\'eg\'e v5.5.0beta. \\
\textbf{E.3 Ontology metrics} \textsc{should} & Number of classes: 21; number of object properties: 11; number of data properties: 4; 0 individuals. \\
\textbf{E.4 Incorporation of other ontologies} \textsc{must} &  \ontoref{sosa, prov, music, event, timeline, foaf}\\
\textbf{E.5 Entity naming convention} \textsc{must} & Entities follows the CamelCase notation. Both datatype and object properties are named as verb senses with mixedCase notation. \\
\textbf{E.6 Identifier generation policy} \textsc{must} & Identifiers of the instances must be generated by the application. \\
\textbf{E.7 Identity metadata policy} \textsc{must} & All entities have an \texttt{rdfs:comment} natural language explanation. \\
\textbf{E.8 Upper ontology} \textsc{must}& See point E.4.\\
\textbf{E.9 Ontology relationships} \textsc{must}& 11 object properties; 4 datatype properties.  \\
\textbf{E.10 Axiom pattern} \textsc{must}& 158 axioms included (of which 68 logical axioms, 40 declaration axioms, 12 \texttt{SubClassOf}, 6 \texttt{EquivalentClass}, 1 \texttt{DisjointClass}, 6 hidden GCI, 5 \texttt{InverseObjectProperty}, 2 \texttt{FunctionalObjectProperty}, 1 Inverse Functional, 4 Asymmetric Object Properties, 4 Irreflexive, 11 ObjectPropertyDomain and Range, 3 Functional DataProperty, 4 DP domain and range, 50 annotation assertions). \\ 
\textbf{E.11 Deferencable URI} \textsc{optional} & It is possible to use deferencable URIs, but no assumption on this is made in the ontology. \\
\toprule
\multicolumn{2}{c}{\textbf{F. Managing change}} \\ \midrule
\textbf{F.1 Sustainability plan} \textsc{must} & Some research projects are being prepared to leverage the ontology. \\
\textbf{F.2 Entity deprecation strategy} \textsc{must}  & Deprecated classes will be labelled as obsolete with a proper annotation property. \\
\textbf{F.3 Versioning policy} \textsc{must} & The IoMusT ontology adopts sequence-based identifiers for its versions with a major number and a minor number, separated by a dot. A novel release featuring only small changes will cause a switch of the minor number, while relevant and/or structural changes affects also the major number.\\
\toprule
\end{tabular}
\end{table*}

\begin{table*}
\centering
\footnotesize
\caption{MIRO Report \cite{matentzoglu2018miro} of the IoMusT Ontology -- Part III of III}
\label{tab:iomust_miro3}
\begin{tabular}{p{.35\textwidth}p{.65\textwidth}}
\toprule
\multicolumn{2}{c}{\textbf{G. Quality assurance}} \\ \midrule
\textbf{G.1 Testing} \textsc{must}& Tests have been made by checking competency questions and formal requirements in the presentation paper. \\
\textbf{G.2 Evaluation} \textsc{must}  & Metrics, and discussions over IoMust ontology evaluation have been discussed in the presentation paper.\\
\textbf{G.3 Examples of use} \textsc{must} & At the moment, only theoretical examples of usage in the presentation paper. \\
\textbf{G.4 Institutional endorsement}  \textsc{optional} & None. \\
\textbf{G.5 Evidence of use} \textsc{must} &  The ontology is still new, but we plan to use it in forthcoming projects. \\
\toprule
& \\
& \\
\multicolumn{2}{c}{\Large IoMust ontology in \faGithub~~~\qrcode{https://fr4ncidir.github.io/IoMusT/}} \\
\centering 
\end{tabular}
\end{table*}