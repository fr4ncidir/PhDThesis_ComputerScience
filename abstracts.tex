\chapter*{Abstract in English}
\addcontentsline{toc}{chapter}{Abstract in English}  
\headletter{I}
n the last two decades the Information Technology changed substantially the life of people all around the World. Just a few years ago, for instance, paper support was needed to exchange all kind of data, while now electronics is indeed the main instrument of communication. This mutation was originally due mostly to the efficiency, while now it is, hopefully, also due to an increased attention towards environmental issues.

Information and data have proved over the time the importance of their role, contributing to a plethora of applications that allow the physical world to interact with mankind by the means of services dispatched pervasively and freely accessible. The Internet is the kernel of such complex setup that is called Internet of Things (IoT).

The IoT inherited from the Internet a chaotic interface. Protocols, conventions, mechanisms are different from an application to the other, and it is difficult and expensive to discover and make applications compatible with one another. From this consideration two exceptional ideas were born, namely the Semantic Web and the Web of Things (WoT). The latter would unify the IoT on an application level shared view, enabling standard discovery mechanisms and definitions. The former, on the other hand, intents to provide the tools to formalize the knowledge contents of the World Wide Web in a simultaneously human and machine understandable way.

This Thesis aims to explore both these two concepts and merge them into the Semantic Web of Things using the best of each. Therefore we hereby propose, describe, evaluate and use two ontologies: the Internet of Musical Things ontology, aiming to outline a semantic description of IoT; and a Semantic WoT ontology, aiming to push further the state of the art of IoT unification and standardization through a dynamic semantic approach.

\chapter*{Abstract in Italiano}
\addcontentsline{toc}{chapter}{Abstract in Italiano}  
\selectlanguage{italian}
\headletter{N}
egli ultimi due decenni le nuove Tecnologie dell'Informazione hanno cambiato radicalmente la vita delle persone in tutto il mondo. Soltanto qualche anno fa, per esempio, lo scambio di informazione era necessariamente effettuato sotto forma cartacea in quasi ogni ambito. Oggi, invece, il mezzo elettronico viene privilegiato sempre più per questioni di efficienza nonch\'e, recentemente, si spera anche per motivi legati alla sostenibilità ambientale.

L'informazione ha dato prova, nel corso del tempo, della sua importanza. Ha contribuito a rendere possibili numerosissime applicazioni in grado di far interagire l'umanità con il mondo fisico attraverso un'astrazione composta da servizi facilmente accessibili e distribuiti ovunque. Internet è il cuore di questo grande sistema chiamato Internet of Things (IoT).

L'IoT ha in comune con Internet la sua interfaccia caotica e la mancanza di ordine. I protocolli, le convenzioni, i meccanismi cambiano da una applicazione all'altra, rendendo difficile e costoso scoprire e creare sistemi compatibili. Da queste considerazioni ormai accettate dalla comunità traggono origine due concetti eccezionali: il Semantic Web e il Web of Things (WoT). Quest'ultimo ha come fine quello di unificare l'IoT ad un livello applicativo condiviso rendendo disponibili definizioni e meccanismi standard per la scoperta dei dispositivi. Il primo, invece, fornisce degli strumenti per formalizzare la conoscenza distribuita nel World Wide Web in modo che sia contemporanemante fruibile all'uomo e alle macchine.

Questa Tesi si accinge ad esplorare i due concetti appena descritti, ed a riunirli usando il meglio di entrambi nel Semantic Web Of Things. Per fare ciò si proporranno, descriveranno, valuteranno ed useranno due ontologie: l'ontologia dell'Internet of Musical Things, che servirà per mostrare una definizione semantica dell'IoT; e l'ontologia del Semantic WoT, il cui scopo è di spingere oltre lo Stato dell'Arte nell'unificazione dell'IoT e nella sua standardizzazione attraverso un approccio semantico e dinamico.

\chapter*{Abstract en Fran\c{c}ais}
\addcontentsline{toc}{chapter}{Abstract en Fran\c{c}ais}  
\selectlanguage{french}
\headletter{L}
es deux dernières décennies ont vu les nouvelles Technologies de l'Information changer de manière radicale la vie des gens partout dans le monde. Il n'y a que quelques années, par exemple, des supports en papier étaient nécessaires pour l'échange des données, alors qu'à présent l'instrument principal est l'électronique. Ce changement était d\^u à l' origine à l'efficacité de la communication. Maintenant, on l'espère, la raison est aussi liée à la tutelle de l'environnement.

Il a été largement démontré que l'information joue un r\^ole essentiel: innombrables applications ont été développées pour connecter le monde physique et l'humanité à travers des services distribués partout et librement accessibles. Internet est au centre de toute cette infrastructure, qui n'est autre que l'Internet des Objets (IoT).

L'IoT et Internet ont en commun leur organisation chaotique. Les protocoles, les conventions, les fonctionnements internes peuvent \^etre très différents d'une application à l'autre: il est souvent difficile et co\^uteux de découvrir et créer des applications compatibles avec le reste des systèmes qui sont à disposition. Ce concept est à l'origine de deux idées exceptionnelles: le Web Sémantique, et le Web des Objets (WoT). Celui-ci a pour but d'unifier l'IoT à un niveau applicatif commun avec des mécanismes de découverte et un vocabulaire standard. Le premier, d'autre part, propose les instruments pour mettre de l'ordre dans la connaissance du World Wide Web, de façon à la rendre à la fois compréhensible à l'\^etre humain, et aux machines.

Cette Thèse explore donc les deux idées à peine présentées, et additionne leurs meilleures qualités pour obtenir le Web des Objets Sémantiques. Ainsi sont proposées, décrites, évaluées et utilisées deux ontologies: celle de l'Internet des Objets Musicaux, pour produire une description sémantique de l'IoT; et celle du WoT Sémantique, qui voudrait avancer l'état de l'art de la recherche sur l'unification de l'IoT de manière sémantique et dynamique.

\selectlanguage{english}